%!TeX program = xelatex
%!TeX encoding = UTF-8 Unicode
\documentclass{article}

\usepackage{empheq}
\usepackage[many]{tcolorbox}
%\tcbuselibrary{skins}

\tcbset{highlight math style={enhanced,
  colframe=black!60!white,arc=4pt,boxrule=1pt,
  }}


\usepackage{enumitem}

\definecolor{vuw}{RGB}{1,81,52}
% \definecolor{vuwfaded}{RGB}{65,134,101}
% \definecolor{vuwfaint}{RGB}{236,244,242}
% 
% 
\definecolor{col2}{RGB}{7,43,78} % blue
\definecolor{col3}{RGB}{157,68,25} % orange
\definecolor{col4}{RGB}{110,1,29} % purple

% math stuff
\usepackage{amssymb,amsmath}
\usepackage{ifxetex,ifluatex}
\usepackage{fixltx2e} % provides \textsubscript
\usepackage{bm} % bold

% font stuff
\usepackage{upquote}
\usepackage{microtype}
% \usefonttheme{professionalfonts} % required for mathspec
\usepackage{fontspec,xltxtra,xunicode}
% \usepackage[MnSymbol]{mathspec}
\defaultfontfeatures{Mapping=tex-text,Scale=MatchUppercase}


\usepackage{sourcesanspro}
% \setsansfont[BoldFont={Source Sans Pro SemiBold}, Numbers={OldStyle}]{Source Sans Pro Light}
% \setsansfont{Latin Modern Sans}
% \setsansfont[BoldFont={Source Serif Pro Bold}, Numbers={OldStyle}]{Source Serif Pro Light}
% \setsansfont[BoldFont={Fira Sans Medium}, Numbers={OldStyle}]{Fira Sans}
\setmonofont[Scale=0.8]{Source Code Pro}

%\usepackage{arev} % sans math fonts

% \usepackage[math-style=TeX]{unicode-math}
% \usepackage[bold-style=ISO]{unicode-math}
% \setmathfont{xits-math.otf}
% \setmathfont{XITS Math}
% \setmathfont{Latin Modern Math}
% \setmathfont{TeX Gyre Schola Math}%horrible
% \setmathfont{TeX Gyre DejaVu Math}
% \setmathfont{Libertinus Math} % OK
% \setmathfont{Latin Modern Math}
% \setmathfont[range=\sum]{Latin Modern Math} % substitute Sum sign
% \setmathfont[range={}]{Latin Modern Math}


%\usepackage{mathabx} % oiint
\usepackage{mathrsfs} %fancy P
\usepackage{fancyvrb}
\usepackage{booktabs}
\usepackage{caption}
\usepackage{url}

\usepackage{graphicx} % already loaded actually
\graphicspath{{./figures/}}
\DeclareGraphicsExtensions{%
    .pdf,.PDF,%
    .png,.PNG,%
    .jpg,.mps,.jpeg,.jbig2,.jb2,.JPG,.JPEG,.JBIG2,.JB2}
\usepackage{grfext}
\PrependGraphicsExtensions*{.png,.pdf}

\makeatletter
\def\maxwidth{\ifdim\Gin@nat@width>\linewidth\linewidth\else\Gin@nat@width\fi}
\def\maxheight{\ifdim\Gin@nat@height>\textheight0.8\textheight\else\Gin@nat@height\fi}
\makeatother
% Scale images if necessary, so that they will not overflow the page
% margins by default, and it is still possible to overwrite the defaults
% using explicit options in \includegraphics[width, height, ...]{}
\setkeys{Gin}{width=\maxwidth,height=\maxheight,keepaspectratio}

\usepackage{hyperref}
\usepackage[normalem]{ulem}
% avoid problems with \sout in headers with hyperref:

\setlength{\parindent}{0pt}
\setlength{\parskip}{6pt plus 2pt minus 1pt}
\setlength{\emergencystretch}{3em}  % prevent overfull lines
% \providecommand{\tightlist}{%
%   \setlength{\itemsep}{0pt}\setlength{\parskip}{0pt}}
\providecommand{\tightlist}{}
% \def\tightlist{} % overwrite pandoc
\setcounter{secnumdepth}{0}
\makeatletter
\makeatother
\makeatletter
\makeatother
\makeatletter
\@ifpackageloaded{caption}{}{\usepackage{caption}}
\AtBeginDocument{%
\ifdefined\contentsname
  \renewcommand*\contentsname{Table of contents}
\else
  \newcommand\contentsname{Table of contents}
\fi
\ifdefined\listfigurename
  \renewcommand*\listfigurename{List of Figures}
\else
  \newcommand\listfigurename{List of Figures}
\fi
\ifdefined\listtablename
  \renewcommand*\listtablename{List of Tables}
\else
  \newcommand\listtablename{List of Tables}
\fi
\ifdefined\figurename
  \renewcommand*\figurename{Figure}
\else
  \newcommand\figurename{Figure}
\fi
\ifdefined\tablename
  \renewcommand*\tablename{Table}
\else
  \newcommand\tablename{Table}
\fi
}
\@ifpackageloaded{float}{}{\usepackage{float}}
\floatstyle{ruled}
\@ifundefined{c@chapter}{\newfloat{codelisting}{h}{lop}}{\newfloat{codelisting}{h}{lop}[chapter]}
\floatname{codelisting}{Listing}
\newcommand*\listoflistings{\listof{codelisting}{List of Listings}}
\makeatother
\makeatletter
\@ifpackageloaded{caption}{}{\usepackage{caption}}
\@ifpackageloaded{subcaption}{}{\usepackage{subcaption}}
\makeatother
\makeatletter
\@ifpackageloaded{tcolorbox}{}{\usepackage[many]{tcolorbox}}
\makeatother
\makeatletter
\@ifundefined{shadecolor}{\definecolor{shadecolor}{rgb}{.97, .97, .97}}
\makeatother
\makeatletter
\makeatother

\title{electromagnetism \& wave optics}
\author{baptiste}
\date{\the\year, T2}


%%%


\newcommand{\task}[2][Exercise:]{\note{\vfill\centering\Large\emph{#1}~#2\vfill}}
% \newcommand{\note}[2]{\note{\vfill\centering\Large\emph{#1}~#2\vfill}}


\newcommand{\epsnot}{\varepsilon_0}
\newcommand{\epsr}{\varepsilon_r}

% hat unit vectors
\newcommand{\vecih}{\mathbf{\hat i}}
\newcommand{\vecjh}{\mathbf{\hat j}}
\newcommand{\veckh}{\mathbf{\hat k}}
\newcommand{\vecyh}{\mathbf{\hat y}}
\newcommand{\veczh}{\mathbf{\hat z}}
\newcommand{\vecdh}{\mathbf{\hat{d}}}
\newcommand{\vecrh}{\mathbf{\hat r}}
\newcommand{\vecsh}{\mathbf{\hat s}}
\newcommand{\vecnh}{\mathbf{\hat n}} 
\newcommand{\vecrhoh}{\mathbf{\hat \rho}}
\newcommand{\vecth}{\mathbf{\hat \theta}}
\newcommand{\vecfh}{\mathbf{\hat \varphi}}
\newcommand{\vecxh}{\mathbf{\hat x}}

\newcommand{\vecx}{\mathbf{x}}
\newcommand{\vecGamma}{\mathbf{\Gamma}}

\newcommand{\vecj}{\mathbf{j}}
\newcommand{\veck}{\mathbf{k}}
\newcommand{\vecv}{\mathbf{v}}
\newcommand{\vecr}{\mathbf{r}}
\newcommand{\vecp}{\mathbf{p}}
\newcommand{\vecd}{\mathbf{d}}

\newcommand{\vecA}{\mathbf{A}}
\newcommand{\vecB}{\mathbf{B}}
\newcommand{\vecC}{\mathbf{C}}
\newcommand{\vecD}{\mathbf{D}}
\newcommand{\vecE}{\mathbf{E}}
\newcommand{\vecF}{\mathbf{F}}
\newcommand{\vecG}{\mathbf{G}}
\newcommand{\vecH}{\mathbf{H}}
\newcommand{\vecJ}{\mathbf{J}}
\newcommand{\vecK}{\mathbf{K}}
\newcommand{\vecP}{\mathbf{P}}
\newcommand{\vecM}{\mathbf{M}}
\newcommand{\vecS}{\mathbf{S}}
\newcommand{\vecV}{\mathbf{V}}
\newcommand{\vecnot}{\mathbf{0}}

% microscopic versions


\newcommand{\vecAm}{\mathbf{a}}
\newcommand{\vecBm}{\mathbf{b}}
\newcommand{\vecCm}{\mathbf{c}}
\newcommand{\vecDm}{\mathbf{d}}
\newcommand{\vecEm}{\mathbf{e}}
\newcommand{\vecFm}{\mathbf{f}}
\newcommand{\vecGm}{\mathbf{g}}
\newcommand{\vecHm}{\mathbf{h}}
\newcommand{\vecJm}{\mathbf{j}}
\newcommand{\vecPm}{\mathbf{p}}
\newcommand{\vecMm}{\mathbf{m}}

% complex phasors

\newcommand{\cx}{\underline{x}}
\newcommand{\cE}{\underline{E}}
\newcommand{\cp}{\underline{p}}
\newcommand{\cP}{\underline{P}}
\newcommand{\cchi}{\underline{\chi}}


\newcommand{\uvecv}{\underline{\mathbf{v}}}
\newcommand{\uvecr}{\underline{\mathbf{r}}}
\newcommand{\uvecp}{\underline{\mathbf{p}}}
\newcommand{\uvecd}{\underline{\mathbf{d}}}
\newcommand{\uvecA}{\underline{\mathbf{A}}}
\newcommand{\uvecB}{\underline{\mathbf{B}}}
\newcommand{\uvecC}{\underline{\mathbf{C}}}
\newcommand{\uvecD}{\underline{\mathbf{D}}}
\newcommand{\uvecE}{\underline{\mathbf{E}}}
\newcommand{\uvecF}{\underline{\mathbf{F}}}
\newcommand{\uvecG}{\underline{\mathbf{G}}}
\newcommand{\uvecH}{\underline{\mathbf{H}}}
\newcommand{\uvecJ}{\underline{\mathbf{J}}}
\newcommand{\uvecP}{\underline{\mathbf{P}}}
\newcommand{\uvecM}{\underline{\mathbf{M}}}


% position vectors
\newcommand{\vecdl}{\mathbf{dl}}
\newcommand{\vecdS}{\mathbf{dS}}
\newcommand{\vecrnot}{\mathbf{r_0}}
\newcommand{\vecrnoth}{\mathbf{\hat r_0}}
\newcommand{\vecrrp}{\mathbf{\vecr-\vecr'}}
\newcommand{\rrp}{\left|\vecrrp\right|}
\newcommand{\overr}{\dfrac{1}{r}}
\newcommand{\rpor}{\dfrac{r'}{r}}
\newcommand{\overrrp}{\dfrac{1}{\rrp}}
\newcommand{\Pleg}{\mathcal{P}}

% blablabla aliases 
\newcommand{\Grad}{\nabla}
\newcommand{\Div}{\nabla\cdot}
\newcommand{\Curl}{\nabla\times}
\newcommand*\Lapl{\mathop{{}\nabla^2}\nolimits}

\newcommand*\Laplacian{\mathop{{}\Delta}\nolimits} %\bigtriangleup
\newcommand*\Dalembertian{\mathop{{}\Box}\nolimits} %\square

\newcommand*\dd{\mathop{}\!\mathrm{d}}
\newcommand*\dS{\mathop{}\!\mathrm{dS}}
\newcommand*\dV{\mathop{}\!\mathrm{dV}}

% bold stuff
\newcommand{\bnabla}{\boldsymbol{\nabla}}
% \DeclareBoldMathCommand\bnabla{\nabla}

% operators
\DeclareMathOperator{\opRe}{Re}
\DeclareMathOperator{\opIm}{Im}


\newcommand{\fullframegraphic}[1]{
    \includegraphics[width=\paperwidth]{#1}
}


%%%
% ,sections,graphics
\usepackage[active,tightpage,displaymath,textmath]{preview}
% \PreviewEnvironment{equation}

\begin{document}



\ifdefined\Shaded\renewenvironment{Shaded}{\begin{tcolorbox}[enhanced, interior hidden, sharp corners, breakable, frame hidden, borderline west={3pt}{0pt}{shadecolor}, boxrule=0pt]}{\end{tcolorbox}}\fi


\begin{frame}{Interface}
\protect\hypertarget{interface}{}
Consider
\(\left(\varepsilon_{1}, \varepsilon_{2}\right) \in \mathbb{R}_{+}^{2}\),
all waves are propagating.

Plane waves:
\(\vecE(\mathbf{r})=\vecE_{0} e^{i \mathbf{k} \cdot \mathbf{r}}\)

The wave-vectors are \[
\mathbf{k}_{i}=\left(\begin{array}{l}
k_{i x} \\
k_{i y} \\
k_{i z}
\end{array}\right) \quad \mathbf{k}_{r}=\left(\begin{array}{l}
k_{r x} \\
k_{r y} \\
k_{r z}
\end{array}\right) \quad \mathbf{k}_{t}=\left(\begin{array}{l}
k_{t x} \\
k_{t y} \\
k_{t z}
\end{array}\right)
\] Electric fields, \[
\vecE_i(\mathbf{r})=\vecE_{0} e^{i \mathbf{k_i} \cdot \mathbf{r}},\vecE_r(\mathbf{r})=r\vecE_{0} e^{i \mathbf{k_r} \cdot \mathbf{r}},\vecE_t(\mathbf{r})=t\vecE_{0} e^{i \mathbf{k_t} \cdot \mathbf{r}}
\]
\end{frame}

\begin{frame}{Snell-Descartes laws}
\protect\hypertarget{snell-descartes-laws}{}
Using the boundary condition for \(\vecE^\parallel\) at \(z=0\) we show
that \(\veck_i,\veck_t,\veck_r\) are in the same plane (``plane of
incidence''), and that \(\veck_\parallel\) is conserved.

\[
\left.e^{i \mathbf{k_i} \cdot \mathbf{r}} + r e^{i \mathbf{k_r} \cdot \mathbf{r}}=t e^{i \mathbf{k_t} \cdot \mathbf{r}}\right|_{z=0}
\] Consider the incident wave in the (xOz) plane (\(k_y=0\)),

\[
e^{i {k_{ix}x}} + r e^{i{k_{rx}x+k_{ry}y}} = t e^{i{k_{tx}x+k_{ty}y}}, \forall (x,y)
\] At \(x=0\): \[
1 + r e^{i{k_{ry}y}} = t e^{i{k_{ty}y}}, \forall y
\] but \(r,t\) do not depend on \((x,y)\) (by assumption: we're looking
for translation-invariant solutions) . The real part reads, \[
1 + r \cos(k_{ry}y) = t \cos({k_{ty}y})
\] By orthogonality of the cosines, this implies that
\(k_{ry}=k_{ty}=0\), and in the general case that the plane of incidence
is conserved.
\end{frame}

\begin{frame}{Snell-Descartes laws (ctd.)}
\protect\hypertarget{snell-descartes-laws-ctd.}{}
Setting \(k_y=0\), \[
e^{i {k_{ix}x}} + r e^{i{k_{rx}x}} = t e^{i{k_{tx}x}}, \forall x
\] multiply by \(e^{-i {k_{ix}x}}\), \[
1 + r e^{i{(k_{rx}-k_{ix})x}} = t e^{i{(k_{tx}-k_{ix})x}}, \forall x
\] which leads to \(k_{rx}=k_{tx}=k_{ix}\): the in-plane momentum is
conserved. Geometrically, for propagating waves,

\[
\theta_i = \theta_r,\quad n_1\sin \theta_i=n_2\sin \theta_t
\]

Note: translational invariance of the problem along the plane implies
conservation of \(k_{x}, k_{y}\) (Noether theorems -- same argument
links time translation symmetry to conservation of energy, rotation
symmetry to conservation of angular momentum, etc.)
\end{frame}

\begin{frame}{k-vectors}
\protect\hypertarget{k-vectors}{}
Dispersion relations: \[
\mathbf{k}_{i} \cdot \mathbf{k}_{i}=k_{0}^{2} \varepsilon_{i} \quad \Rightarrow k_{i}=k_{0} \sqrt{\varepsilon_{i}}=k_{0} n_{i}
\] Explicitly, \[
\left\{\begin{array}{l}k_{i x}=n_{1} k_{0} \sin \theta_{i} \\ k_{r x}=n_{1} k_{0} \sin \theta_{r} \\ k_{t x}=n_{2} k_{0} \sin \theta_{t}\end{array}\right.
\] leading to \[
\left\{\begin{array}{c}\theta_{i}=\theta_{r} \\ n_{1} \sin \theta_{i}=n_{2} \sin \theta_{t}\end{array}\right.
\] (used in geometrical optics)
\end{frame}

\begin{frame}{Interface: Fresnel coefficients}
\protect\hypertarget{interface-fresnel-coefficients}{}
Plane of incidence:
\((x 0 z) \Rightarrow \mathbf{k}_{1}=k_{1 x} \mathbf{e}_{x}+k_{1 z} \mathbf{e}_{z}\)

Boundary conditions at \(z=0\), \(k_{i x}=k_{r x}=k_{t x} \equiv k_{x}\)

Dispersion relations:
\(\mathbf{k} \cdot \mathbf{k}=\varepsilon k_{0}^{2}\) \[
\Rightarrow \left\{\begin{array}{l}
k_{x}^{2}+k_{i z}^{2}=\varepsilon_{1} k_{0}^{2} \quad \Rightarrow k_{i z}=\pm \sqrt{\varepsilon_{1} k_{0}^{2}-k_{x}^{2}}=-k_{r z} \\
k_{x}^{2}+k_{r z}^{2}=\varepsilon_{1} k_{0}^{2} \\
k_{x}^{2}+k_{t z}^{2}=\varepsilon_{2} k_{0}^{2}
\end{array}\right.
\]
\end{frame}

\begin{frame}{TE case}
\protect\hypertarget{te-case}{}
For TE-polarised light, the continuity of \(E^y\) reads, {[} E\_i + E\_r
= E\_t {]} which yields, \[
  1+r =t.
\] The continuity of \(H^x\) gives us another equation. \(H^x\) is
obtained from \(\veck\times\vecE = \omega\vecB = \omega\mu\vecH\), \[
\omega\mu H_x = k_z E_y
\] leading to: \[
\frac{k_{iz}}{\mu_1}E_{iy} + \frac{k_{rz}}{\mu_1}E_{ry} = \frac{k_{tz}}{\mu_2}E_{ty}
\] and noting that \(k_{rz} = -k_{iz}\), \[
\frac{k_{iz}}{\mu_1} - \frac{k_{iz}}{\mu_1}r = \frac{k_{tz}}{\mu_2}t,
\] \[
1 - r = \frac{\mu_1}{\mu_2}\frac{k_{tz}}{k_{iz}}t
\]
\end{frame}

\begin{frame}{TE case (ctd.)}
\protect\hypertarget{te-case-ctd.}{}
We can summarize the two continuity relations in the following system,

\[\left\{
\begin{aligned}
1 + r &= t\\
1 - r &= \frac{\mu_1}{\mu_2}\frac{k_{tz}}{k_{iz}}t
\end{aligned}\right.
\] Solving for \(r\) and \(t\) yields the result, \[
t_{12}^s=\frac{2\mu_2 k_{z1}}{\mu_2 k_{z1}+\mu_1k_{z2}},\qquad r_{12}^s=\frac{\mu_2 k_{z1}-\mu_1k_{z2}}{\mu_2 k_{z1}+\mu_1k_{z2}}.
\] \[
t_{12}^s=\frac{2}{1+K_s},\qquad r_{12}^s=\frac{1 - K_s}{1 + K_s},\qquad K_s:=\frac{\mu_1}{\mu_2}\frac{k_{z2}}{k_{z1}}
\]
\end{frame}

\begin{frame}{Fresnel coefficients}
\protect\hypertarget{fresnel-coefficients}{}
To summarize, for a single interface from 1 to 2 with normal along the
\(z\) direction, the Fresnel coefficients read, \[
    \begin{aligned}
    \rho_{12}^p & =\frac{\varepsilon_2 k_{z1}-\varepsilon_1k_{z2}}{\varepsilon_2 k_{z1}+\varepsilon_1k_{z2}} = \frac{1-K^p}{1+K^p},%
    & {}& r_{12}^s=\frac{\mu_2 k_{z1}-\mu_1k_{z2}}{\mu_2 k_{z1}+\mu_1k_{z2}} = \frac{1-K^s}{1+K^s}\\
    \tau_{12}^p &=\frac{2\varepsilon_2 k_{z1}}{\varepsilon_2 k_{z1}+\varepsilon_1k_{z2}} = \frac{2}{1+K^p},%
    &{}& t_{12}^s=\frac{2\mu_2 k_{z1}}{\mu_2 k_{z1}+\mu_1k_{z2}} = \frac{2}{1+K^s}
    \end{aligned}
\] with \(K^p:= \frac{\varepsilon_1 k_{z2}}{\varepsilon_2 k_{z1}}\),
\(K^s:= \frac{ k_{z2}}{ k_{z1}}\) (often we will assume \(\mu=1\)).

Note that, \[
    r_{ij}=-r_{ji}.
\] For propagating incident and transmitted waves, equivalent
expressions can also be given in terms of the angle of incidence
\(\theta\) : \[
r^{p}=\frac{n_{2} \cos \theta-n_{1} \sqrt{1-\frac{n_{1}^{2}}{n_{2}^{2}} \sin ^{2} \theta}}{n_{2} \cos \theta+n_{1} \sqrt{1-\frac{n_{1}^{2}}{n_{2}^{2}} \sin ^{2} \theta}} 
\]
\end{frame}

\end{document}
