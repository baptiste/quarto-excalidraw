%!TeX program = xelatex
%!TeX encoding = UTF-8 Unicode
\documentclass{article}

\usepackage{empheq}
\usepackage[many]{tcolorbox}
%\tcbuselibrary{skins}

\tcbset{highlight math style={enhanced,
  colframe=black!60!white,arc=4pt,boxrule=1pt,
  }}


\usepackage{enumitem}

\definecolor{vuw}{RGB}{1,81,52}
% \definecolor{vuwfaded}{RGB}{65,134,101}
% \definecolor{vuwfaint}{RGB}{236,244,242}
% 
% 
\definecolor{col2}{RGB}{7,43,78} % blue
\definecolor{col3}{RGB}{157,68,25} % orange
\definecolor{col4}{RGB}{110,1,29} % purple

% math stuff
\usepackage{amssymb,amsmath}
\usepackage{ifxetex,ifluatex}
\usepackage{fixltx2e} % provides \textsubscript
\usepackage{bm} % bold

% font stuff
\usepackage{upquote}
\usepackage{microtype}
% \usefonttheme{professionalfonts} % required for mathspec
\usepackage{fontspec,xltxtra,xunicode}
% \usepackage[MnSymbol]{mathspec}
\defaultfontfeatures{Mapping=tex-text,Scale=MatchUppercase}


\usepackage{sourcesanspro}
% \setsansfont[BoldFont={Source Sans Pro SemiBold}, Numbers={OldStyle}]{Source Sans Pro Light}
% \setsansfont{Latin Modern Sans}
% \setsansfont[BoldFont={Source Serif Pro Bold}, Numbers={OldStyle}]{Source Serif Pro Light}
% \setsansfont[BoldFont={Fira Sans Medium}, Numbers={OldStyle}]{Fira Sans}
\setmonofont[Scale=0.8]{Source Code Pro}

%\usepackage{arev} % sans math fonts

% \usepackage[math-style=TeX]{unicode-math}
% \usepackage[bold-style=ISO]{unicode-math}
% \setmathfont{xits-math.otf}
% \setmathfont{XITS Math}
% \setmathfont{Latin Modern Math}
% \setmathfont{TeX Gyre Schola Math}%horrible
% \setmathfont{TeX Gyre DejaVu Math}
% \setmathfont{Libertinus Math} % OK
% \setmathfont{Latin Modern Math}
% \setmathfont[range=\sum]{Latin Modern Math} % substitute Sum sign
% \setmathfont[range={}]{Latin Modern Math}


%\usepackage{mathabx} % oiint
\usepackage{mathrsfs} %fancy P
\usepackage{fancyvrb}
\usepackage{booktabs}
\usepackage{caption}
\usepackage{url}

\usepackage{graphicx} % already loaded actually
\graphicspath{{./figures/}}
\DeclareGraphicsExtensions{%
    .pdf,.PDF,%
    .png,.PNG,%
    .jpg,.mps,.jpeg,.jbig2,.jb2,.JPG,.JPEG,.JBIG2,.JB2}
\usepackage{grfext}
\PrependGraphicsExtensions*{.png,.pdf}

\makeatletter
\def\maxwidth{\ifdim\Gin@nat@width>\linewidth\linewidth\else\Gin@nat@width\fi}
\def\maxheight{\ifdim\Gin@nat@height>\textheight0.8\textheight\else\Gin@nat@height\fi}
\makeatother
% Scale images if necessary, so that they will not overflow the page
% margins by default, and it is still possible to overwrite the defaults
% using explicit options in \includegraphics[width, height, ...]{}
\setkeys{Gin}{width=\maxwidth,height=\maxheight,keepaspectratio}

\usepackage{hyperref}
\usepackage[normalem]{ulem}
% avoid problems with \sout in headers with hyperref:

\setlength{\parindent}{0pt}
\setlength{\parskip}{6pt plus 2pt minus 1pt}
\setlength{\emergencystretch}{3em}  % prevent overfull lines
% \providecommand{\tightlist}{%
%   \setlength{\itemsep}{0pt}\setlength{\parskip}{0pt}}
\providecommand{\tightlist}{}
% \def\tightlist{} % overwrite pandoc
\setcounter{secnumdepth}{0}
\makeatletter
\makeatother
\makeatletter
\makeatother
\makeatletter
\@ifpackageloaded{caption}{}{\usepackage{caption}}
\AtBeginDocument{%
\ifdefined\contentsname
  \renewcommand*\contentsname{Table of contents}
\else
  \newcommand\contentsname{Table of contents}
\fi
\ifdefined\listfigurename
  \renewcommand*\listfigurename{List of Figures}
\else
  \newcommand\listfigurename{List of Figures}
\fi
\ifdefined\listtablename
  \renewcommand*\listtablename{List of Tables}
\else
  \newcommand\listtablename{List of Tables}
\fi
\ifdefined\figurename
  \renewcommand*\figurename{Figure}
\else
  \newcommand\figurename{Figure}
\fi
\ifdefined\tablename
  \renewcommand*\tablename{Table}
\else
  \newcommand\tablename{Table}
\fi
}
\@ifpackageloaded{float}{}{\usepackage{float}}
\floatstyle{ruled}
\@ifundefined{c@chapter}{\newfloat{codelisting}{h}{lop}}{\newfloat{codelisting}{h}{lop}[chapter]}
\floatname{codelisting}{Listing}
\newcommand*\listoflistings{\listof{codelisting}{List of Listings}}
\makeatother
\makeatletter
\@ifpackageloaded{caption}{}{\usepackage{caption}}
\@ifpackageloaded{subcaption}{}{\usepackage{subcaption}}
\makeatother
\makeatletter
\@ifpackageloaded{tcolorbox}{}{\usepackage[many]{tcolorbox}}
\makeatother
\makeatletter
\@ifundefined{shadecolor}{\definecolor{shadecolor}{rgb}{.97, .97, .97}}
\makeatother
\makeatletter
\makeatother

\title{electromagnetism \& wave optics}
\author{baptiste}
\date{\the\year, T2}


%%%


\newcommand{\task}[2][Exercise:]{\note{\vfill\centering\Large\emph{#1}~#2\vfill}}
% \newcommand{\note}[2]{\note{\vfill\centering\Large\emph{#1}~#2\vfill}}


\newcommand{\epsnot}{\varepsilon_0}
\newcommand{\epsr}{\varepsilon_r}

% hat unit vectors
\newcommand{\vecih}{\mathbf{\hat i}}
\newcommand{\vecjh}{\mathbf{\hat j}}
\newcommand{\veckh}{\mathbf{\hat k}}
\newcommand{\vecyh}{\mathbf{\hat y}}
\newcommand{\veczh}{\mathbf{\hat z}}
\newcommand{\vecdh}{\mathbf{\hat{d}}}
\newcommand{\vecrh}{\mathbf{\hat r}}
\newcommand{\vecsh}{\mathbf{\hat s}}
\newcommand{\vecnh}{\mathbf{\hat n}} 
\newcommand{\vecrhoh}{\mathbf{\hat \rho}}
\newcommand{\vecth}{\mathbf{\hat \theta}}
\newcommand{\vecfh}{\mathbf{\hat \varphi}}
\newcommand{\vecxh}{\mathbf{\hat x}}

\newcommand{\vecx}{\mathbf{x}}
\newcommand{\vecGamma}{\mathbf{\Gamma}}

\newcommand{\vecj}{\mathbf{j}}
\newcommand{\veck}{\mathbf{k}}
\newcommand{\vecv}{\mathbf{v}}
\newcommand{\vecr}{\mathbf{r}}
\newcommand{\vecp}{\mathbf{p}}
\newcommand{\vecd}{\mathbf{d}}

\newcommand{\vecA}{\mathbf{A}}
\newcommand{\vecB}{\mathbf{B}}
\newcommand{\vecC}{\mathbf{C}}
\newcommand{\vecD}{\mathbf{D}}
\newcommand{\vecE}{\mathbf{E}}
\newcommand{\vecF}{\mathbf{F}}
\newcommand{\vecG}{\mathbf{G}}
\newcommand{\vecH}{\mathbf{H}}
\newcommand{\vecJ}{\mathbf{J}}
\newcommand{\vecK}{\mathbf{K}}
\newcommand{\vecP}{\mathbf{P}}
\newcommand{\vecM}{\mathbf{M}}
\newcommand{\vecS}{\mathbf{S}}
\newcommand{\vecV}{\mathbf{V}}
\newcommand{\vecnot}{\mathbf{0}}

% microscopic versions


\newcommand{\vecAm}{\mathbf{a}}
\newcommand{\vecBm}{\mathbf{b}}
\newcommand{\vecCm}{\mathbf{c}}
\newcommand{\vecDm}{\mathbf{d}}
\newcommand{\vecEm}{\mathbf{e}}
\newcommand{\vecFm}{\mathbf{f}}
\newcommand{\vecGm}{\mathbf{g}}
\newcommand{\vecHm}{\mathbf{h}}
\newcommand{\vecJm}{\mathbf{j}}
\newcommand{\vecPm}{\mathbf{p}}
\newcommand{\vecMm}{\mathbf{m}}

% complex phasors

\newcommand{\cx}{\underline{x}}
\newcommand{\cE}{\underline{E}}
\newcommand{\cp}{\underline{p}}
\newcommand{\cP}{\underline{P}}
\newcommand{\cchi}{\underline{\chi}}


\newcommand{\uvecv}{\underline{\mathbf{v}}}
\newcommand{\uvecr}{\underline{\mathbf{r}}}
\newcommand{\uvecp}{\underline{\mathbf{p}}}
\newcommand{\uvecd}{\underline{\mathbf{d}}}
\newcommand{\uvecA}{\underline{\mathbf{A}}}
\newcommand{\uvecB}{\underline{\mathbf{B}}}
\newcommand{\uvecC}{\underline{\mathbf{C}}}
\newcommand{\uvecD}{\underline{\mathbf{D}}}
\newcommand{\uvecE}{\underline{\mathbf{E}}}
\newcommand{\uvecF}{\underline{\mathbf{F}}}
\newcommand{\uvecG}{\underline{\mathbf{G}}}
\newcommand{\uvecH}{\underline{\mathbf{H}}}
\newcommand{\uvecJ}{\underline{\mathbf{J}}}
\newcommand{\uvecP}{\underline{\mathbf{P}}}
\newcommand{\uvecM}{\underline{\mathbf{M}}}


% position vectors
\newcommand{\vecdl}{\mathbf{dl}}
\newcommand{\vecdS}{\mathbf{dS}}
\newcommand{\vecrnot}{\mathbf{r_0}}
\newcommand{\vecrnoth}{\mathbf{\hat r_0}}
\newcommand{\vecrrp}{\mathbf{\vecr-\vecr'}}
\newcommand{\rrp}{\left|\vecrrp\right|}
\newcommand{\overr}{\dfrac{1}{r}}
\newcommand{\rpor}{\dfrac{r'}{r}}
\newcommand{\overrrp}{\dfrac{1}{\rrp}}
\newcommand{\Pleg}{\mathcal{P}}

% blablabla aliases 
\newcommand{\Grad}{\nabla}
\newcommand{\Div}{\nabla\cdot}
\newcommand{\Curl}{\nabla\times}
\newcommand*\Lapl{\mathop{{}\nabla^2}\nolimits}

\newcommand*\Laplacian{\mathop{{}\Delta}\nolimits} %\bigtriangleup
\newcommand*\Dalembertian{\mathop{{}\Box}\nolimits} %\square

\newcommand*\dd{\mathop{}\!\mathrm{d}}
\newcommand*\dS{\mathop{}\!\mathrm{dS}}
\newcommand*\dV{\mathop{}\!\mathrm{dV}}

% bold stuff
\newcommand{\bnabla}{\boldsymbol{\nabla}}
% \DeclareBoldMathCommand\bnabla{\nabla}

% operators
\DeclareMathOperator{\opRe}{Re}
\DeclareMathOperator{\opIm}{Im}


\newcommand{\fullframegraphic}[1]{
    \includegraphics[width=\paperwidth]{#1}
}


%%%
% ,sections,graphics
\usepackage[active,tightpage,displaymath,textmath]{preview}
% \PreviewEnvironment{equation}

\begin{document}



\ifdefined\Shaded\renewenvironment{Shaded}{\begin{tcolorbox}[boxrule=0pt, borderline west={3pt}{0pt}{shadecolor}, frame hidden, enhanced, interior hidden, sharp corners, breakable]}{\end{tcolorbox}}\fi


\begin{frame}{Inconsistency for time-varying currents}
\protect\hypertarget{inconsistency-for-time-varying-currents}{}
By time of James Clerk Maxwell (mid-19th century):

\[\begin{aligned}
\Div\vecE& =\rho / \varepsilon_0\qquad\text{(Gau\ss)}\\
\Div\vecB&=0\qquad\text{(no magnetic monopoles)}\\
\Curl\vecE&=-\partial_t \vecB\qquad\text{(Faraday)}\\
\Curl\vecB&=\mu_0 \vecJ\qquad\text{(Ampère)}
\end{aligned}
\]

\hrule

BUT\dots~an inconsistency arises, as \(\Div(\Curl\vecV)\equiv 0\) for
any vector \(\vecV\): \[
\Div(\Curl\vecE)=-\partial_t (\Div\vecB)=0 \qquad(\text{all good})
\] Yet, \[
\Div(\Curl\vecB)=\mu_0 \Div\vecJ \neq 0 \qquad(\text{problem!})
\]

\(\Div\vecJ = -\partial_t \rho\), so \(\Div\vecJ = 0\) only for steady
currents (magnetostatics).
\end{frame}

\begin{frame}{Maxwell's contribution}
\protect\hypertarget{maxwells-contribution}{}
Maxwell realised that the last equation needed an additional term.
Applying continuity of current/charge: \[
\begin{aligned}
\Div\vecJ &=-\frac{\partial \rho}{\partial t} \\
&=-\frac{\partial}{\partial t}\left(\varepsilon_0\Div\vecE\right)=-\Div\left(\varepsilon_0 \frac{\partial \vecE}{\partial t}\right)
\end{aligned}
\] If we add \(\mu_0 \varepsilon_0 \frac{\partial \vecE}{\partial t}\)
to the current \(\vecJ\), the inconsistency disappears and the last curl
equation becomes: \[
\Curl\vecB=\mu_0 \vecJ+\mu_0 \varepsilon_0 \frac{\partial \vecE}{\partial t} \quad \text { (Ampère's law with Maxwell's correction) }
\] \(\varepsilon_0 \frac{\partial \vecE}{\partial t}\) is called the
``\emph{displacement current density}''.
\end{frame}

\begin{frame}{Electrodynamics}
\protect\hypertarget{electrodynamics}{}
Our starting point: the electromagnetic field obeys Maxwell's equations:
\[
\begin{aligned}
\Div\vecD &= \rho_f\\
\Div\vecB &= 0\\
\Curl\vecE &= -\frac{\partial\vecB}{\partial t}\\
\Curl\vecH &= \frac{\partial\vecD}{\partial t} + \vecJ_f
\end{aligned}
\] Lorentz force: \[
\vecF = q\left(\vecE + \vecv \times \vecB\right)
\] + \emph{constitutive relations} (\(\sigma, \varepsilon, \mu,\dots\))
and \emph{boundary conditions}
\end{frame}

\begin{frame}{Remark on constitutive relations}
\protect\hypertarget{remark-on-constitutive-relations}{}
Material responses are very diverse (anisotropic, gyrotropic, nonlinear,
nonlocal, hysteresis, \dots). We'll only discuss the most common
approximations,

\begin{itemize}
\tightlist
\item
  \emph{Linear approximation}: \[
  \begin{array}{l}
  \mathbf{P}(\mathbf{r}, t)=\varepsilon_{0}\left(\overline{\bar{\chi}}_{e}^{(1)} \mathbf{E}(\mathbf{r}, t)+\overline{\bar{\chi}}_{e}^{(2)} \mathbf{E}^{2}(\mathbf{r}, t)+\overline{\bar{\chi}}_{e}^{(3)} \mathbf{E}^{3}(\mathbf{r}, t)+\ldots\right) \\
  \mathbf{P}(\mathbf{r}, t)=\varepsilon_{0} \overline{\bar{\chi}}_{e} \mathbf{E}(\mathbf{r}, t)
  \end{array}
  \]
\item
  \emph{Isotropic medium}: \(\chi_e\) is a scalar (\(\mathbf{P}\) and
  \(\mathbf{E}\) have the same orientation)
\item
  \emph{Non-locality in space}: \(\vecP\) depends on the electric field
  in the vicinity of a point
  \(\mathbf{P}(\mathbf{r}, t)=\varepsilon_{0} \int d \mathbf{r}^{\prime} \chi\left(\mathbf{r}^{\prime}-\mathbf{r}, t\right) \mathbf{E}\left(\mathbf{r}^{\prime}, t\right)\)
  If we use spatial Fourier Transform, this convolution product simply
  becomes:
  \(\underline{\mathbf{P}}(\mathbf{k}, t)=\varepsilon_{0} \underline{\chi_{e}}(\mathbf{k}, t) \underline{\mathbf{E}}(\mathbf{k}, t)\)
  \(\to \varepsilon(\veck,\omega)\) (cf Kittel, etc.)
\end{itemize}
\end{frame}

\end{document}
